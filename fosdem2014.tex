\documentclass[aspectratio=169]{beamer}
\usepackage[english,russian]{babel}
\usepackage[utf8]{inputenc}
\usepackage{verbatim}
\usepackage{graphicx}
\usepackage{pgfpages}
\usepackage{ulem}
\usepackage{float}
\usepackage{amsmath}

\setbeameroption{hide notes}

\setbeamercolor{title}{fg=white}
\setbeamercolor{author}{fg=white}
\setbeamercolor{normal text}{fg=black}
\setbeamercolor{frametitle}{fg=black}
\setbeamercolor{item}{fg=red}
\setbeamercolor{block title}{fg=red}
\setbeamercolor{section in toc}{fg=red}
\setbeamercolor{footline}{fg=white}
\setbeamercolor{title in head/foot}{fg=white,bg=black}

\setbeamertemplate{navigation symbols}{}
\setbeamertemplate{headline}{
    \includegraphics[height=1mm, width=\paperwidth]{wg-headline.png}
}

\setbeamertemplate{footline}{
    \begin{beamercolorbox}[ht=1.2em]{title in head/foot}
        {\footnotesize \hspace{1em}\inserttitle, \insertshortauthor}
    \end{beamercolorbox}
}

\begin{document}

\title{FOSDEM 2014}
\author{Maksim Melnikau}
\date{}

{
\title{
    \\
    {\huge FOSDEM 2014}
    \\
}

\usebackgroundtemplate{\includegraphics[width=\paperwidth]{wg-end.jpg}}
\begin{frame}[plain]{}
    \titlepage
\end{frame}
}

\usebackgroundtemplate{\includegraphics[width=\paperwidth]{wg-bg.jpg}}

\begin{frame}{FOSDEM}
  \begin{itemize}
  \item I'm at \#fosdem 2014! \url{http://pic.twitter.com/cGep6vrjTE}
  \item K level1, I think it is 2\% of all attendences here \url{http://pic.twitter.com/e0XAVnaND3}
  \item 1st day is over, a lot of tweeting without laptop charging, and powertop says that I have 1 hour 40 minutes more - very nice!
  \item FOSDEM is over, thank you, see you next time \url{http://pic.twitter.com/LDRGMYury1}
  \item 300+ hours video recorded, enough to see 1h/day till next \#fosdem
 \end{itemize}
\end{frame}

\begin{frame}{How Find and Fix Million Grammar and Style Errors in Wikipedia (1/2)}
  \begin{itemize}
  \item \url{http://pic.twitter.com/Xa3X3qhXlW}
  \item Wikipedia uses languagetool (Java) to find errors
  \item languagetool finds errors, but explanation sometimes wrong
  \item languagetool - 10ms per sentence, 37k articles checked, 1 mln errors is projection
  \item All sessions will be recorded ---except the java dev room they are too cool
  \item there is so many falsa alarms - difficult text extraction, math language, non-English terms
  \item languagetool false error examples: 68000 assembler (suggest: assemblers), if a is algebraic over K(suggest: an)
  \item LanguageTool: the next step after spell checking, LGPL, 10 regular commiters, Java+XML
  \item languagetool: plain text => sentences => words => find part-of-speech and base form => analyzed sentences against error patterns
  \item languagetool patterns are easy to contribute in XML format, no java skills required
  \item languagetool supports many different languages including russian and belarusian
  \end{itemize}
\end{frame}

\begin{frame}{How Find and Fix Million Grammar and Style Errors in Wikipedia (2/2)}
  \begin{itemize}
  \item grammar is set of rules that describe how valid words, sentences, and texts looks like
  \item English wasn't made for being parsed
  \item "Sorry for my bed English" grammatically is fine...
  \item errors find, but how to *fix the million* Wikipedia errors?
  \item fix errors from recent changes http://community.languagetool.org/feedMatches/list?lang=en …, fix only changed part
  \item Must check OpenNLP for machine learning. \#fosdem
  \item languagetool wish: make style and grammar checking ubiquitous
  \item Belarusian haven't maintainer in languagetool team
  \item no need to stick to spell checking today - more powerful checks are available
  \end{itemize}
\end{frame}

\begin{frame}{kdbus, Lennart Poettering (1/2)}
  \begin{itemize}
  \item \url{http://pic.twitter.com/or7GRd4qIK}
  \item D-Dbus is powerful IPC: method call transactions, signals, properties, broadcasting, discovery, introspection, policy, activation...
  \item D-Bus ... security, monitoring, expose APIs, File Description passing, Language agnostic, Network transparency, no trust required...
  \item D-Bus has limitations: suitable only for control, not payload; inefficient; not available in early boot, initrd; baroque codebase...
  \item if you try to solve problem with XML, you have two problems, about dbus in kdbus talk
  \item but still, D-Bus is fantastic, solves real problems
  \end{itemize}
\end{frame}
  
\begin{frame}{kdbus, Lennart Poettering (2/2)}
  \begin{itemize}
  \item kdbus suitable for large data (GiB!), zero-copy, optionally reusable; implicit timestamping; always available; no XML...
  \item kdbus overview: receiver buffer, single copy to dst(s), method call window, name registr ...
  \item memfds: zero copy, sealing, 512K zero copy is faster than single copy (a bit like Android ashmem)
  \item 2 previous tries to get d-bus in kernel grandiosly failed
  \end{itemize}
\end{frame}
 
\begin{frame}{miracast on Linux}
  \begin{itemize}
  \item \url{http://pic.twitter.com/59Nu5lWk9n}
  \item miracast: HDMI over IP over Wifi
  \item ieee 802.11; wifi-p2p => wifi direct; wifi-display => miracast
  \item miracast: P2P transport setup, ip link auto discovery, A/V streams
  \item mirascast: many Linux wifi drivers not working (b43, brcmac, rtl818x, ath5k), some supposed to work (ath9k,brcmfmac, iwl-mvm)...
  \item known to work: iwl+mwm + intel wifi 7260 + wpa\_supplicant: git-78f79 ...
  \item HDMI over IP is RTSP + RTP + h264 + audio + mpeg2-TS
  \item Additional Features: PTP, HDCP, UIBC, split-sink
  \end{itemize}
\end{frame}

\begin{frame}{Sailfish and Jolla (1/2)}
  \begin{itemize}
  \item \url{http://pic.twitter.com/BDyv8WiN39}
  \item half people at sailfish talk have jolla device already!
  \item jolla: no factory images, but recovery mode; fastboot; abillity to unlock bootloader and flash own kernels; full root available
  \item sailfishos: systemd, gcc, btrfs, gstreamer, wayland, qt5
  \item libhybris - leverage existing Android hardware adaptation
  \item https://together.jolla.com/questions/  - joll's idea how to make users involve to contriubute
  \item \#jolla contribute to everything  except: artwork/trademark and L\&F UI, 3rd party closed source drivers, some NDA stuff ...
  \item contribute to sailfishos: contribute to nemo, mer, and a lot of upstream projects!
  \end{itemize}
\end{frame}  
  
\begin{frame}{Sailfish and Jolla (2/2)}
  \begin{itemize}
  \item libhybris - port Android/bionic linker to glibc environment, and it works; allows use existing hardware for non-android OSs
  \item it seems ridiculous to load glibc and bionic to address space of process but works for almost all cases
  \item android\_dlopen("libEGL.so") - we could build (glibc) libEGL.so and libGLESv2.so wrappers that accessed the android ones!
  \item libhybris today used by Jolla/SailfishOS, Intel/Tizen, Canonical/Ubuntu
  \end{itemize}
\end{frame}  
  
\begin{frame}{Fedora.Next}
  \begin{itemize}
  \item \url{http://pic.twitter.com/EVINqplhDK}
  \item Fedora.Next split to Workstation, Server, Cloud
  \item Fedora Workstation - graphical user environment for Student, Independent Developer, Small Company Developer, Developer in large Org
  \item Fedora Server - Headless "pet" server, Server Roles, IaaS Host, Stable platform for critical infrastructure
  \item Fedora Cloud - cloud image "cattle" server, scale-out, packaged images for public clouds
  \item some people in room dislike that Fedora.NEXT is smth, which was "designed" behind closed doors
  \item Fedora has so many infrastructure problems: bugs, reviews, build system, etc.
  \end{itemize}
\end{frame}

\begin{frame}{FOSDEM network, NAT64 and DNS64 (1/2)}
  \begin{itemize}
  \item \url{http://pic.twitter.com/JQwXYYaw5z}
  \item NAT64 Statistics Total active translations: 23636 
  \item but too many people escaped to fosdem-dualstack :( )
  \item So, I'm switched to ipv6 \#fosdem ESSID ... It is my first time ever when I use ipv6 ... \url{http://pic.twitter.com/KO1nCoCwHr}
  \item IPv4 has run out, IPv5 never made it to public use, so IPv6
  \item there was a war in begging of IPv6: 64bit vs unlimited!
  \item clients, content, carriers, applications, hardware - Mexican standoff - nobody want to do first step
  \item World IPv6 day - lets turn it on and see what breaks
  \item google, facebook, yahoo, youtube, netflix, akamai and many more run ipv6 today
  \item different countries (main providers) enables ipv6 one by one - France, Germany, Belgium etc
  \end{itemize}
\end{frame}
  
\begin{frame}{FOSDEM network, NAT64 and DNS64 (2/2)}
  \begin{itemize}
  \item - 5000+ hackers which could test, debug and fix ipv6 problems
  \item if you run NAT anyway - why not unable IPv6 and use NAT64 and DNS64 ?!
  \item we can hide a complete legacy internet in a /96!
  \item My twit at \#fosdem ipv6 talk, I'm famous now! :) \url{http://pic.twitter.com/EEqM0XKIQM}
  \item nexus could not get ipv6 only address
  \item FOSDEM'14 is the first general-purpose conference which has ipv6 network by default
  \end{itemize}
\end{frame}

\begin{frame}{KDE Connect}
  \begin{itemize}
  \item \url{http://pic.twitter.com/Z5ZaXSvzRT}
  \item KDE Connect - fuse your devices as mush as possible and desirable
  \item KDE Connect protocol: json based, medium abstracted, easy extended, easy implemented
  \item KDE Connect: Notifications, Actions, Battery, MPRIS2, Send files and Urls, Clipboard synchronization, Encryption
  \item Connect: Qt => libconnect => Server => Plugins => DBus => Plasma, KCM, Apps
  \end{itemize}
\end{frame}  
  
\begin{frame}{GPU Offload on Wayland}
  \begin{itemize}
  \item \url{http://pic.twitter.com/BIUK0CmUp8}
  \item render-nodes - Allow to render without authentication to DRM master(but without some functionality)
  \item 1080p screen buffer with 60fps ~ 480MB/s, PCI express is 4GB/s, thunderbolt ~ 1GB/s
  \item tiling - special pixel ordering optimized to exploit local spatial coherence - good for performance
  \item GPU offload with X DRI2: DDX per device/provider, configure with xrandr
  \item Two displays: A and B, two cards: 1 connected to A, 2 connected to B - classic nvidia optimus layout
  \item wayland gpu offload: shutdown the dedicated GPU when unneeded works now
  \item XWayland: wlglamor, Xserver linked to Wayland compositor - no need for gpu offloading
  \end{itemize}
\end{frame} 

\begin{frame}{Wine User Experience}
  \begin{itemize}
  \item \url{http://pic.twitter.com/WGR6nkEhuQ}
  \item once a year somebody writes at wine forum what "everything is work, and you are rock!"
  \item \#ubuntu still ships 1.4.x wine version, why?!
  \item common problem when you answer to user question if user hides, you don't know why: does everything work, user give up or died ...
  \end{itemize}
\end{frame} 
 
\begin{frame}{Performance of Wine and Graphical Drivers (1/2)}
  \begin{itemize}
  \item \url{http://pic.twitter.com/jRFI2OxRqT}
  \item wine performance becomes a bit faster on windows last year (yes, wine works on windows too)
  \item R300g+wine become a bit slower with reason of some unknown regression, but R600g got greate improvements
  \item nvidia legacy is unchanged, so quite silent time from user point of view => but it was a lot of work inside
  \item wine multithread command stream - move most d3d work into separate thread => better CPU utilization => 2x performance (in theory)
  \item easy synchronization in multi-threaded games, even bigger performance gains, ~3x in CoD 4:MW, btw Windows does the same thing
  \item GeForce 460 GTX, wine performance a 40\% lower than in windows for UT2004
  \end{itemize}
\end{frame}  
  
\begin{frame}{Performance of Wine and Graphical Drivers (2/2)}
  \begin{itemize}
  \item wine CSMT improvements - some games faster on wine than on windows
  \item CSMT brings better performance mostly on fast systems like 460gtx + i7 ...
  \item drivers don't like to be called from two threads without looking, even with separate contexts
  \item CSMT and Nvidia's threaded opt - essentially the same thing, differences is in synchronization
  \item CSMT wine next steps: upstream, improve data streaming, reduce draw overhead in wine, wine performance outside d3d ...
  \item wine could have a big problems running on wayland natively - many windows apps rely on window posistions for example
  \end{itemize}
\end{frame}  
  
\begin{frame}{Persistant Storage (1/2)}
  \begin{itemize}
  \item \url{http://pic.twitter.com/5TOmumYiiw}
  \item file system performance: maximize throughput or latency? target embedded, power consumption or performance?
  \item high bandwidth has been the traditional focus - backup, streaming video, etc
  \item SSD's made life more complicated - not too painful at first, plaggued in
  \item PCI-e SSD Devices Turn up the Heat - opened a lot of other bottlenecks in Linux Storage Stack - 1 million IOPS/device
  \item a single file system is easy for users and applications, and can perform better than multiple file systems
  \item wow! fsck which works more than week!
  \item persistent memory - a variety of new technologies are coming from multiple vendors - Linux need to be (mostly) technology neutral
  \item SNIA - Storage Network Industry Association, Working Group on NVM.
  \end{itemize}
\end{frame} 
  
\begin{frame}{Persistant Storage (2/2)}
  \begin{itemize}
  \item SMR Drive Write Bands - sequential write only
  \item Host is aware of SMR topology at some layer
  \item Open source drives emerging for PCI-e cards, open source drivers should become more popular than closed vendor one
  \item SMR and PM together - interesting workload for out future
  \item for x86\_64 machines, normally block size limit is 4k, storage hardware often have very large internal blocks - 65k
  \item Persistent Storage - you will never ever wait for storage anymore, CPU will be bottleneck
  \end{itemize}
\end{frame} 

\begin{frame}{Concurrent Programming Made Simple - Transaction Memory (1/2)}
  \begin{itemize}
  \item \url{http://pic.twitter.com/loaFWPhMFI}
  \item shared memory (synchronization) + Transactions = Transaction memory (TM)
  \item TM is programming abstraction - allow programmers to declare which code sequences are atomic
  \item TM is still rather new - standartization for C/C++ started 5 years go, GCC has support since 4.7, HW implementations - haswell
  \item \_\_transaction\_atomic \{ if (x<10) y++; \} - code in atomic transactions must be transaction-safe
  \item transactions extend the C11/C++1 memory model - all transactions totally ordered
  \item TM supports modular programming - programmers don't need to manage association between shared data and synchronization metadata
  \end{itemize}
\end{frame}
  
\begin{frame}{Concurrent Programming Made Simple - Transaction Memory (2/2)}
  \begin{itemize}
  \item GCC implementation: compiler - ensure atomicity guarantee (at compile time!) - find all transaction safe code
  \item GCC implementation: TM runtime library (libitm) => enforces atomicity o transactions at runtime (contains SW-only implementation)
  \item performance: it's a tool, not magic - *useful balance* between easy-to-use and performance, but implementations are wip
  \item single-thread performance: STM slower than sequential, HTM equals. In multi-thread both STM and HTM scales well
  \item TM, use it: gcc -fgnu-tm, report bugs and dive into libitm / GCC
  \item Transaction Memory as Distributed Transaction Memory
  \item eventually consistency - is not consistency at all!
  \end{itemize}
\end{frame}

{
\setbeamertemplate{footline}{}
\setbeamercolor{frametitle}{fg=white}
\setbeamercolor{normal text}{fg=white}
\setbeamercolor{block title}{fg=white}
\setbeamercolor{block body}{fg=red}

\usebackgroundtemplate{\includegraphics[height=\paperheight]{wg-end.jpg}}
\begin{frame}{Thank You. Questions}
    \begin{block}{Maksim Melnikau}
    \par \url{mailto:m\_melnikau@wargaming.net}
    \par \url{https://plus.google.com/+MaksimMelnikau}
    \par \url{https://twitter.com/max\_posedon}
    \par \url{http://wargaming.com}
    \end{block}
\end{frame}
}

\end{document}
